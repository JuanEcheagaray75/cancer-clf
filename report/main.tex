\documentclass[journal]{IEEEtran}                                                          % if you need a4paper
%\documentclass[a4paper, 10pt, conference]{ieeeconf}      % Use this line for a4
                                                         % paper
\IEEEoverridecommandlockouts        % This command is only
% needed if you want to
                                    % use the \thanks command
%\overrideIEEEmargins
% See the \addtolength command later in the file to balance the column lengths
% on the last page of the document
% The following packages can be found on http:\\www.ctan.org
\usepackage{graphics} % for pdf, bitmapped graphics files
\usepackage{rotating} % rotate figures
\usepackage{epsfig} % for postscript graphics files
%\usepackage{mathptmx} % assumes new font selection scheme installed
%\usepackage{times} % assumes new font selection scheme installed
\usepackage{amsmath}
\usepackage{amssymb}
\usepackage[spanish]{babel}
\usepackage{cite}

\usepackage{atbegshi} % erase first blank page
\usepackage{hyperref}

\AtBeginDocument{\AtBeginShipoutNext{\AtBeginShipoutDiscard}}

\title{\LARGE \bf Proyecto: Minería de datos}

%%%%%%%%%%%%%%%%%%%%%% AUTHORS %%%%%%%%%%%%%%%%%%%%%%%%%%%%%%%%%%%%%%%%5
\author{Juan Pablo Echeagaray González, Emily Rebeca Méndez Cruz, Grace Aviance Silva Arostegui}% <-this % stops 
\begin{document}

    \thanks{Juan Pablo Echeagaray González, Emily Rebeca Méndez Cruz, Grace Aviance Silva Arostegui pertencen al Tec de Monterrey campus Monterrey, N.L. C.P. 64849, Mexico {\tt\small}}

    \maketitle

    \thispagestyle{empty}
    \pagestyle{empty}
    %%%%%%%%%%%%%%%%%%%%%%%%%%%%%%%%%%%%%%%%%%%%%%%%%%%%%%%%%%%%%%%%%%%%%%%%%%%%%
    \begin{abstract}
        Referencia perrona de este libro \cite{geron-2019}
    \end{abstract}

    \begin{IEEEkeywords} 
    Data Science, Machine Learning, Data Analysis
    \end{IEEEkeywords}

    %%%%%%%%%%%%%%%%%%%%%%%%%%%%%%%%%%%%%%%%%%%%%%%%%%%%%%%%%%%%%%%%%%%%%%%%%%%%%%%%
    \section{Introducción} \label{introduction}

    \section{Créditos} \label{credits}
       
        \begin{itemize}
            \item Juan Pablo Echeagaray González - A00830646
            \item Emily Rebeca Méndez Cruz
            \item Grace Aviance Silva Aróstegui
        \end{itemize}

    \section{Modelos de Machine Learning} \label{modelos} % [Juan Pablo]

        \subsection{Árbol de decisión} \label{decision-tree}
            \cite{sci-kit-learn-no-dateA}
        \subsection{Support Vector Machine (SVM)} \label{svm}
            % \cite{sci-kit-learn-svm}
            Haciendo uso de Python y de la librería \emph{Folium} hemos visualizado una muestra de los últimos pedidos registrados en la base de datos, Fig; cada punto azul representa un cliente, y el punto negro dentro del mapa es donde se encuentra el CEDIS.

        \subsection{Red Neuronal} \label{neural-network}
            % uwu \cite{team-2022} \cite{team-2022}
            Después de inspeccionar el mapa generado hemos notado que hay algunos puntos que parecen tener datos geográficos erróneos, descartar la entrega a estos clientes es algo inaceptable, así que una de las siguientes tareas en el proyecto será desarrollar un método de limpieza efectivo que ayude a mejorar la información geográfica que obtengamos de cada punto.

        \subsection{Regresión Logística} \label{logistic}
            % uwu \cite{sci-kit-learn-log}
    \section{Resultados} \label{resultados}

    \section{Conclusiones} \label{conclusiones}
        
        \subsection{Áreas de mejora} \label{improvements}

        \subsection{Modelo seleccionado} \label{selected-model}

    \section{Reflexiones} \label{thoughts}
    
    \appendices
    
    \section{Datos}
        Los datos usados en este proyecto pueden descargarse \href{https://www.kaggle.com/code/ravaliraj/risk-classification-of-cervical-cancer}{aquí}

    \section{Código}
        El código desarrollado se encuentra en el siguiente \href{https://github.com/JuanEcheagaray75/cancer-clf}{repositorio}
    \section{Evidencias de trabajo en equipo}
    \bibliographystyle{IEEEtran}
    \bibliography{references.bib}

\end{document}